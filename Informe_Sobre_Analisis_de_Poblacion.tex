%===================================================================================
% JORNADA CIENTÍFICA ESTUDIANTIL - MATCOM, UH
%===================================================================================
% Esta plantilla ha sido diseñada para ser usada en los artículos de la
% Jornada Científica Estudiantil de MatCom.
%
% Por favor, siga las instrucciones de esta plantilla y rellene en las secciones
% correspondientes.
%
% NOTA: Necesitará el archivo 'jcematcom.sty' en la misma carpeta donde esté este
%       archivo para poder utilizar esta plantila.
%===================================================================================



%===================================================================================
% PREÁMBULO
%-----------------------------------------------------------------------------------
\documentclass[a4paper,10pt,twocolumn]{article}

%===================================================================================
% Paquetes
%-----------------------------------------------------------------------------------
\usepackage{amsmath}
\usepackage{amsfonts}
\usepackage{amssymb}
\usepackage{jcematcom}
\usepackage[utf8]{inputenc}
\usepackage{listings}
\usepackage[pdftex]{hyperref}
\usepackage{caption}
\usepackage{subcaption}
%-----------------------------------------------------------------------------------
% Configuración
%-----------------------------------------------------------------------------------
\hypersetup{colorlinks,%
	    citecolor=black,%
	    filecolor=black,%
	    linkcolor=black,%
	    urlcolor=blue}

%===================================================================================



%===================================================================================
% Presentacion
%-----------------------------------------------------------------------------------
% Título
%-----------------------------------------------------------------------------------
\title{Documento de Ejemplo para la Jornada Científica Estudiantil}

%-----------------------------------------------------------------------------------
% Autores
%-----------------------------------------------------------------------------------
\author{\\
\name Autor Uno \email \href{mailto:a.uno@lab.matcom.uh.cu}{a.uno@lab.matcom.uh.cu}
	\\ \addr Grupo B612 \AND
\name Autor Dos \email \href{mailto:a.dos@lab.matcom.uh.cu}{a.dos@lab.matcom.uh.cu}
  \\ \addr Grupo B612}

%-----------------------------------------------------------------------------------
% Tutores
%-----------------------------------------------------------------------------------
\tutors{\\
Dr. Tutor Uno, \emph{Centro} \\
Lic. Tutor Dos, \emph{Centro}}

%-----------------------------------------------------------------------------------
% Headings
%-----------------------------------------------------------------------------------
\jcematcomheading{\the\year}{1-\pageref{end}}{A. Uno, A. Dos}

%-----------------------------------------------------------------------------------
\ShortHeadings{Ejemplo JCE}{Autores}
%===================================================================================



%===================================================================================
% DOCUMENTO
%-----------------------------------------------------------------------------------
\begin{document}

%-----------------------------------------------------------------------------------
% NO BORRAR ESTA LINEA!
%-----------------------------------------------------------------------------------
\twocolumn[
%-----------------------------------------------------------------------------------

\maketitle

%===================================================================================
% Resumen y Abstract
%-----------------------------------------------------------------------------------
\selectlanguage{spanish} % Para producir el documento en Español

%-----------------------------------------------------------------------------------
% Resumen en Español
%-----------------------------------------------------------------------------------
\begin{abstract}

	El resumen en español debe constar de $100$ a $200$ palabras y presentar de forma
	clara y concisa el contenido fundamental del artículo.

\end{abstract}

%-----------------------------------------------------------------------------------
% English Abstract
%-----------------------------------------------------------------------------------
\vspace{0.5cm}

\begin{enabstract}

  The English abstract must have have $100$ to $200$ words, and present 
  the essentials of the article content in a clear and concise form.

\end{enabstract}

%-----------------------------------------------------------------------------------
% Palabras clave
%-----------------------------------------------------------------------------------
\begin{keywords}
	Separadas,
	Por,
	Comas.
\end{keywords}

%-----------------------------------------------------------------------------------
% Temas
%-----------------------------------------------------------------------------------
\begin{topics}
	Tema, Subtema.
\end{topics}


%-----------------------------------------------------------------------------------
% NO BORRAR ESTAS LINEAS!
%-----------------------------------------------------------------------------------
\vspace{0.8cm}
]
%-----------------------------------------------------------------------------------


%===================================================================================

%===================================================================================
% Resumen Extendido
%-----------------------------------------------------------------------------------
\section{Resumen Extendido}\label{sec:intro}
%-----------------------------------------------------------------------------------
  Este documento proporciona una plantilla para confeccionar el artículo de un 
  trabajo a presentar en la Jornada Científica Estudiantil en el Evento Académico. El artículo no excederá 
  de 2 páginas y SOLO contará con esta sección. Además no debe sobrepasar las 500 palabras. En esta sección debe incluir una presentación del dominio de su 
  problema, los objetivos y motivaciones fundamentales de su investigación así como 
  un resumen del estado del arte al respecto. Puede aprovechar para tratar las propuestas y métodos de solución empleados, así como los resultados obtenidos, pero todo a modo de resumen.
  
%-----------------------------------------------------------------------------------
Para producir listas enumeradas, utilice el siguiente estilo:
\begin{enumerate}
	\item Primer Elemento
	\item Segundo Elemento
	%
	\begin {enumerate}
	\item {Segundo Elemento - Subítem Uno}
	\item {Segundo Elemento - Subítem Dos}
	\end {enumerate}
	%
\end{enumerate}

%-----------------------------------------------------------------------------------
Para producir descripciones, use el siguiente estilo:

%-----------------------------------------------------------------------------------
\begin{description}
	\item [Primer Elemento] con su respectiva descripción.
	\item [Segundo Elemento] también con su respectiva descripción.
\end{description}

%-----------------------------------------------------------------------------------
Para producir cuerpos flotantes (figuras o tablas), asegúrese de numerar
y etiquetar correctamente cada figura. Las referencias a las figuras deben
estar correctamente etiquetadas. Por ejemplo, véase la Fig. \ref{fig:ex}\ldots

\begin{figure}[h!]%
	\begin{center}
		\begin{tabular}{|c|c|c|} \hline
			& Método 1 	& Método 2 	\\ \hline
			A 			&  			&  			\\ \hline
			B			& 			& 			\\ \hline
			C 			& 			&  			\\ \hline
		\end{tabular}
		\caption{Figura de ejemplo. Recuerde especificar el origen de los datos que se muestran. \label{fig:ex}}
	\end{center}
\end{figure}
%-----------------------------------------------------------------------------------

Para producir código fuente, envuélvalo en una figura flotante y
etiquételo correctamente. Por ejemplo, en la Fig. \ref{fig:code}
se muestra un código bastante conocido\ldots

% Configuración de Listings
\lstset{keywordstyle=\color{blue}, basicstyle=\small}

\begin{figure}[h!]%
	\begin{lstlisting}[language=c]%
		
     int main(int argc, char** argv)
     {
         // Imprimiendo "Hola Mundo".
         printf("Hello, World");
     }
		
	\end{lstlisting}
	\caption{Código fuente de ejemplo.\label{fig:code}}
\end{figure}

%===================================================================================


\label{end}

\end{document}

%===================================================================================
